\documentclass{beamer}
\mode<presentation>
{
  \usetheme{metropolis}      % or try Darmstadt, Madrid, Warsaw, ...
  \usecolortheme{default} % or try albatross, beaver, crane, ...
  \usefonttheme{default}  % or try serif, structurebold, ...
  \setbeamertemplate{navigation symbols}{}
  \setbeamertemplate{caption}[numbered]
} 

\usepackage{bookmark}
\usepackage[english]{babel}
\usepackage[utf8x]{inputenc}
\usepackage{framed}
\usepackage{amsmath,amsfonts,amssymb}
\usepackage{hyperref}
\usepackage{verbatim}
\usepackage{dsfont}
\usepackage{ulem}

\def\s{\phantom{-}}

\title{Introduction to Poisson Regression with Robust Standard Errors - Part 4}
\author{Christopher Liu, Siyu Zhao, Zheming Yu, Runnan Liu, Shiming Zhang}
\institute{San Diego State University, Stats 610}
\date{11/29/2018}

\begin{document}



\begin{frame}[plain]

	\titlepage
	
	\footnote{a}

\end{frame}


	
\begin{frame}[fragile]\frametitle{Objectives of the Video}

	\begin{itemize}
	
		\item Discuss the motivation for the Poisson Regression model with Robust Standard Errors
			
		\item Understand the difference between this model and the "regular" Poisson Regression model		
				
		\item Understand the interpretation of the model's parameters
	
	\end{itemize}	

\end{frame}



\begin{frame}[fragile]\frametitle{Intro to Poisson Regression with Robust Standard Errors}
	
	What is Poisson Regression with \textbf{Robust Standard Errors}?

	$\rightarrow$ Modified Poisson Regression model that can work with outcomes that are binary

\end{frame}



\begin{frame}[fragile]\frametitle{The Motivation}

	Many real-world scenarios can be modeled with binary regression
	
	\begin{itemize}
	
		\item Health-related problems such as association of disease and certain factors
		
		\item Widely used in fields such as Epidemiology and Public Health
		
		\item Allows us to estimate the "Relative Risk"
	
	\end{itemize}
	
\end{frame}



\begin{frame}[fragile]\frametitle{The Framework of the Model}
	
	The framework for the regression model is exactly the same as the "regular" Poisson Regression Model:
	$$ log(\lambda_i|X_i) = \beta_0 + \beta_1 x_{i1} + ... + \beta_p x_{ip} = X_i \beta $$  

	\begin{itemize}
	
		\item Log link function for our GLM
		
		\item $\beta$'s are our predictor variables	
		
		\item $\lambda_i$ is the expected value of our output variable
	
	\end{itemize}

\end{frame}



\begin{frame}[fragile]\frametitle{Adjustments to Make}

	Why do we need to modify the model to work with binary data?

	Main problem is with the Poisson assumption of 	
	$$ E(Y_i) = Var(Y_i) $$ 
	\begin{itemize}

		\item With binomial data, Poisson regression usually underestimates variance of data
		
		\item For binomial case, the mean is usually greater than the variance
		
	\end{itemize}
	
\end{frame}



\begin{frame}[fragile]\frametitle{Adjustments to Make}
	
	The only adjustment to make:	
	
	\begin{itemize}
		
		\item The standard errors of the estimated predictors, $\hat{\beta}$, are replaced with "robust" standard errors (also known as Huber-White Standard Errors) 
		
		$$ Var(\hat{\beta}) = (X^{T}X)^{-1}X^{T} \Sigma X(X^{T}X)^{-1} $$
		
		where $\Sigma$ is the covariance matrix of the residuals			
		
	\end{itemize}	
	
\end{frame}



\begin{frame}[fragile]\frametitle{Implementing the Robust Standard Errors}
	
	$$ Var(\hat{\beta}) = (X^{T}X)^{-1}X^{T} \Sigma X(X^{T}X)^{-1} $$
		
	\begin{itemize}
	
		\item The "sandwich" name comes from the appearance of the equation
		
		\item Simple to implement/acquire through software such as R with the "sandwich" library
		
	\end{itemize}
	
\end{frame}



\begin{frame}[fragile]\frametitle{Differences in the Models}

	The estimates of the predictor variables are acquired in the same way with the Likelihood Function. As a result:	
	\begin{itemize}

		\item The values of the $\hat{\beta}$'s are the exact same
		
		\item Their estimated standard errors will differ due to the use of the Sandwich Estimator
		
		\item Their interpretations change to match the data being analyzed

	\end{itemize}
	
\end{frame}



\begin{frame}[fragile]\frametitle{Interpretation of the Predictors}

	Note the derivation is exactly the same:
	
	$$ log(\lambda_i|x+1) - log(\lambda_i|x) $$
	$$ = \beta_0 + \beta_1 (x + 1) - (\beta_0 + \beta_1 x)$$ 
	$$ = \beta_1 $$
	$$ \implies \frac{(\lambda_i|x+1)}{(\lambda_i|x)} = e^{\beta_1} $$
	
\end{frame}



\begin{frame}[fragile]\frametitle{Interpretation of the Predictors}

	But we are now working with binomial data
	\begin{itemize}
	
		\item The expected value of the output is a probability
		
		\item The expected value of "regular" Poisson Regression outputs is a count 
	
	\end{itemize}
	
\end{frame}



\begin{frame}[fragile]\frametitle{Parameter Interpretation Example}
	
	Ex. Assume our model gives: $$ \frac{(\lambda_i|x+1)}{(\lambda_i|x)} = 1.15 $$
	
	\begin{itemize}
		
		\item An increase in $x$ by $1$ unit, with every other predictor held constant, increases the \textbf{probability of the outcome} by $15$ percent
	
	\end{itemize}
	
	Compare this with the interpretation for the "regular" Poisson Regression model
	
\end{frame}



\begin{frame}[fragile]\frametitle{In the next video}
	
	In the next video:	
	
	\begin{itemize}
	
		\item Tutorial of how to apply the Poisson Regression model with Robust Standard Errors to real data with binary outcomes in R
		
		\item Working example of interpreting the output in terms of the theory just discussed
		
	\end{itemize}
	
\end{frame}



\end{document}