\documentclass{beamer}
\mode<presentation>
{
  \usetheme{metropolis}      % or try Darmstadt, Madrid, Warsaw, ...
  \usecolortheme{default} % or try albatross, beaver, crane, ...
  \usefonttheme{default}  % or try serif, structurebold, ...
  \setbeamertemplate{navigation symbols}{}
  \setbeamertemplate{caption}[numbered]
} 

\usepackage{bookmark}
\usepackage[english]{babel}
\usepackage[utf8x]{inputenc}
\usepackage{framed}
\usepackage{amsmath,amsfonts,amssymb}
\usepackage{hyperref}
\usepackage{verbatim}
\usepackage{dsfont}
\usepackage{ulem}

\def\s{\phantom{-}}

\title{Poisson Regression with Robust Standard Errors}
\author{Christopher Liu, Siyu Zhao, Zheming Yu, Runnan Liu, Shiming Zhang}
\institute{San Diego State University, Stats 610}
\date{11/29/2018}

\begin{document}

	

\begin{frame}[plain]

	\titlepage
	
	\footnote{a}

\end{frame}



\begin{frame}[fragile]\frametitle{The Poisson Distribution}

	A random variable Y is said to have a Poisson distribution with parameter $\lambda$ if its probability is given by the probability mass function
	$$ Pr(Y = y) = \frac{e^{-\lambda}\lambda^{y}}{y!} $$
	for $\lambda > 0$ and y = 0, 1, 2, ... 
 
	The mean and variance of this distribution can be shown to be
	$$ E(Y) = Var(Y) = \lambda $$ 

\end{frame}



\begin{frame}[fragile]\frametitle{Introduction to Poisson Regression}
	In Poisson Regression:
	
	\begin{itemize}
	
		\item Model used when the desired response variable, $Y_{i}$, is a count (eg. Number of vehicle accidents per year, number of visits to a website over a certain time span, etc)
		\item We can also have the response variable be $Y_{i}$/t, the rate at which the event happens with t being an interval representing time, space, or some other grouping of interest

	\end{itemize}

\end{frame}



\begin{frame}[fragile]\frametitle{Introduction to Poisson Regression}

	\begin{itemize}
		
		\item The regression model with the log link function:
 		$$ log(\lambda_i) = \beta_0 + \beta_1 x_{i1} + ... + \beta_p x_{ip} = X_i \beta $$  		
 		where $ E(Y_i) = \lambda_i = e^{X_i \beta}$

		\item Predictor variables are estimated by maximizing the likelihood function:
		$$ L(\beta) = \prod_{i=1}^{n} f(Y_i) = \prod_{i=1}^{n} \frac{ e^{-\lambda_i} \lambda_i^{Y_i} }{Y_i!} $$
		
	\end{itemize}
	
\end{frame}



\begin{frame}[fragile]\frametitle{Poisson Regression with Robust Standard Errors}
	
	Our focus, however, is to discuss \textbf{Poisson Regression with Robust Standard Errors}

	\begin{itemize}
	
		\item Modified Poisson Regression that can work with response variables with binary outcomes
		
		\item Addresses problems with overdispersion

	\end{itemize}
	
\end{frame}



\begin{frame}[fragile]\frametitle{Poisson Regression with Robust Standard Errors}

	Main problem is with the Poisson assumption of 	
	$$ E(Y_i) = Var(Y_i) $$ 

	\begin{itemize}

		\item With binomial data, Poisson regression usually underestimates variance of data
		
		\item We need a way to address this problem when making inferences

	\end{itemize}
	
\end{frame}



\begin{frame}[fragile]\frametitle{Poisson Regression with Robust Standard Errors}
	
	The only adjustments to make:	
	
	\begin{itemize}
	
		\item We keep the maximum likelihood estimates of $ \beta $
		
		\item The standard errors, however, are replaced with "robust" standard errors from the sandwich estimator

	\end{itemize}
	
\end{frame}



\begin{frame}[fragile]\frametitle{Poisson Regression with Robust Standard Errors}
	
	
\end{frame}



\end{document}