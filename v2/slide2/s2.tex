\documentclass{beamer}
\mode<presentation>
{
  \usetheme{metropolis}      % or try Darmstadt, Madrid, Warsaw, ...
  \usecolortheme{default} % or try albatross, beaver, crane, ...
  \usefonttheme{default}  % or try serif, structurebold, ...
  \setbeamertemplate{navigation symbols}{}
  \setbeamertemplate{caption}[numbered]
} 

\usepackage{bookmark}
\usepackage[english]{babel}
\usepackage[utf8x]{inputenc}
\usepackage{framed}
\usepackage{amsmath,amsfonts,amssymb}
\usepackage{hyperref}
\usepackage{verbatim}
\usepackage{dsfont}
\usepackage{ulem}

\def\s{\phantom{-}}

\title{Introduction to Poisson Regression with Robust Standard Errors - Part 2}
\author{Christopher Liu, Siyu Zhao, Zheming Yu, Runnan Liu, Shiming Zhang}
\institute{San Diego State University, Stats 610}
\date{11/29/2018}

\begin{document}

	

\begin{frame}[plain]

	\titlepage
	
	\footnote{a}

\end{frame}



\begin{frame}[fragile]\frametitle{Objectives of the Video}

	\begin{itemize}
	
		\item Develop a theoretical basis for the Poisson Regression Model
		
		\item Understand the interpretation of the model's parameters
		
		\item Understand the uses and limitations of this specific regression model
	
	\end{itemize}	

\end{frame}



\begin{frame}[fragile]\frametitle{Recap of the Poisson Distribution}

	Recall $Y \sim Poisson(\lambda)$ if:
	$$ Pr(Y = y) = \frac{e^{-\lambda}\lambda^{y}}{y!} $$
	for $\lambda > 0$ and y = 0, 1, 2, ... 
 
	Also its mean and variance is given by:
	$$ E(Y) = Var(Y) = \lambda $$ 

\end{frame}



\begin{frame}[fragile]\frametitle{Introduction to the Poisson Regression Model}
	The Poisson Regression model:
	
	\begin{itemize}
	
		\item Used to model situations where the desired output, $Y_{i}$, is a count of something
		
		\item $Y_i$ is a random variable with a Poisson distribution
		
		\item Examples include the number of vehicle accidents per year or the number of visits to a website over a certain time span

	\end{itemize}

\end{frame}



\begin{frame}[fragile]\frametitle{Generalized Linear Models}

	The Poisson Regression Model is a \textbf{Generalized Linear Model} (GLM) with the general form:
	$$ f(\beta,X) = \beta_0 + \beta_1 x_{i1} + ... + \beta_p x_{ip} = X_i \beta $$

	$f(\beta,X)$ is called the \textbf{Link Function} and relates the expected value of the output variable, $Y_i$, to the linear equation
	
\end{frame}



\begin{frame}[fragile]\frametitle{Poisson Regression as a Generalized Linear Model}

	The Poisson Regression model uses the log link function:
 	$$ log(\lambda_i|X_i) = \beta_0 + \beta_1 x_{i1} + ... + \beta_p x_{ip} = X_i \beta $$  		
 	where $ log(\lambda_i|X_i) = log[E(Y_i|X_i)] $
		
	
\end{frame}



\begin{frame}[fragile]\frametitle{Estimating the Predictor Variables}

	The predictor variables for the regression equation are estimated by maximizing the likelihood function:
		$$ L(\beta) = \prod_{i=1}^{n} f(Y_i) = \prod_{i=1}^{n} \frac{ e^{-\lambda_i} \lambda_i^{Y_i} }{Y_i!} $$
		
	We call them the maximum likelihood estimates and denote as $\hat{\beta}$
	
\end{frame}



\begin{frame}[fragile]\frametitle{Interpreting the Parameters}	
	
	Given our regression model
	$$ log(\lambda_i|X_i) = \beta_0 + \beta_1 x_{i1} + ... + \beta_p x_{ip} = X_i \beta $$  
	with the predictor variables $\beta_0, \beta_1, ..., \beta_p$
	
	\begin{itemize}
	
		\item How do we interpret the $\beta_i$'s with respect to our data?	
	
	\end{itemize}
	
\end{frame}



\begin{frame}[fragile]\frametitle{Interpreting the Parameters}	
	
	Take the simple case: $$ log(\lambda_i|x) = \beta_0 + \beta_1 x $$

	Consider the difference between the mean response given $(x+1)$ and the mean response given $x$
	
\end{frame}



\begin{frame}[fragile]\frametitle{Interpreting the Parameters}
	Consider the difference between the mean response, $\lambda_i$ given $(x+1)$ and the mean response, $\lambda_i$, given $x$:
	$$ log(\lambda_i|x+1) - log(\lambda_i|x) $$
	$$ = \beta_0 + \beta_1 (x + 1) - (\beta_0 + \beta_1 x)$$ 
	$$ = \beta_1 $$
	$$ \implies \frac{(\lambda_i|x+1)}{(\lambda_i|x)} = e^{\beta_1} $$
\end{frame}



\begin{frame}[fragile]\frametitle{Parameter Interpretation Example}
	
	Ex. Assume our model gives: $$ \frac{(\lambda_i|x+1)}{(\lambda_i|x)} = 1.15 $$
	
	\begin{itemize}
	
		\item $(\lambda_i|x+1)$ is $15$ percent greater than $(\lambda_i|x)$
		
		\item An increase in $x$ by $1$ unit, with every other predictor held constant, increases the expected value/count by $.15$
	
	\end{itemize}
	
\end{frame}



\begin{frame}[fragile]\frametitle{In the next video}
	
	In the next video:	
	
	\begin{itemize}
	
		\item Tutorial of how to apply the Poisson Regression model to real data in R
		
		\item Working example of interpreting the output in terms of the theory just discussed
		
	\end{itemize}
	
\end{frame}



\end{document}