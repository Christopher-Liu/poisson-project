\documentclass{beamer}
\mode<presentation>
{
  \usetheme{metropolis}      % or try Darmstadt, Madrid, Warsaw, ...
  \usecolortheme{default} % or try albatross, beaver, crane, ...
  \usefonttheme{default}  % or try serif, structurebold, ...
  \setbeamertemplate{navigation symbols}{}
  \setbeamertemplate{caption}[numbered]
} 

\usepackage{bookmark}
\usepackage[english]{babel}
\usepackage[utf8x]{inputenc}
\usepackage{framed}
\usepackage{amsmath,amsfonts,amssymb}
\usepackage{hyperref}
\usepackage{verbatim}
\usepackage{dsfont}
\usepackage{ulem}

\def\s{\phantom{-}}

\title{Introduction to Poisson Regression with Robust Standard Errors - Part 1}
\author{Christopher Liu, Siyu Zhao, Zheming Yu, Runnan Liu, Shiming Zhang}
\institute{San Diego State University, Stats 610}
\date{11/29/2018}

\begin{document}

	

\begin{frame}[plain]

	\titlepage
	
	\footnote{a}

\end{frame}



\begin{frame}[fragile]\frametitle{Content Map of the Video Series}
	
	Video 1: Introduction to the Poisson distribution
	
	Video 2: Introduction to Poisson Regression for count data
		
	Video 3: Working example in R of the Poisson Regression model for count data		
		
	Video 4: Introduction to Poisson Regression with Robust Standard Errors for binary outcome data
		
	Video 5: Working example in R of the Poisson Regression model with Robust Standard Errors

\end{frame}



\begin{frame}[fragile]\frametitle{Goals of this Video}

	The goal is to cover the Poisson distribution
	
	$\rightarrow$ If you have prior knowledge of the Poisson distribution, feel free to move ahead to the second video in which we will begin introducing the Poisson Regression model

\end{frame}



\begin{frame}[fragile]\frametitle{The Poisson Distribution}

	A random variable $Y$ is said to have a Poisson distribution with parameter $\lambda$ if its probability is given by the probability mass function
	$$ Pr(Y = y) = \frac{e^{-\lambda}\lambda^{y}}{y!} $$
	for $\lambda > 0$ and y = 0, 1, 2, ...  
 

\end{frame}



\begin{frame}[fragile]\frametitle{Expected Value and Variance}

	For a random variable $Y \sim Poisson(\lambda)$, the expected value (mean) and variance can be shown to be
	$$ E(Y) = Var(Y) = \lambda $$ 
	
	Note: $E(Y)$ and $Var(Y)$ are equal to each other- this will be important to consider when we talk about regression!

\end{frame}



\begin{frame}[fragile]\frametitle{Uses for the Poisson Distribution}

	The Poisson distribution is used to describe random variables that give the count of some event per unit time, space, or other grouping
	
	For example:
	\begin{itemize}

		\item $W =$ number of emergency phone calls per square kilometer in a region
	
		\item $X =$ number of accidents at an intersection per week
		
		\item $Y =$ number of births at a hospital per day
	
	\end{itemize}

\end{frame}



\begin{frame}[fragile]\frametitle{In the Next Video}

	In the next video:
	\begin{itemize}

		\item Discuss how this distribution helps us model data
		
		\item Learn the theory behind the Poisson Regression Model
	
	\end{itemize}

\end{frame}


\end{document}