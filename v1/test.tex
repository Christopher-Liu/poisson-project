\documentclass{beamer}
%
% Choose how your presentation looks.
%
% For more themes, color themes and font themes, see:
% http://deic.uab.es/~iblanes/beamer_gallery/index_by_theme.html
%
\mode<presentation>
{
  \usetheme{metropolis}      % or try Darmstadt, Madrid, Warsaw, ...
  \usecolortheme{default} % or try albatross, beaver, crane, ...
  \usefonttheme{default}  % or try serif, structurebold, ...
  \setbeamertemplate{navigation symbols}{}
  \setbeamertemplate{caption}[numbered]
} 

\usepackage{bookmark}
\usepackage[english]{babel}
\usepackage[utf8x]{inputenc}
\usepackage{framed}
\usepackage{amsmath,amsfonts,amssymb}
\usepackage{hyperref}
\usepackage{verbatim}
\usepackage{dsfont}
\usepackage{ulem}

\def\s{\phantom{-}}

\title{Robust Poisson Regression}
\author{Siyu Zhao, Zheming Yu, Christopher Liu, Runnan Liu, Shiming Zhang}
\institute{San Diego State University, Stats 610}
\date{10/13/2018}

\begin{document}

	

\begin{frame}[plain]

	\titlepage
	
	\footnote{a}

\end{frame}



\begin{frame}[fragile]\frametitle{The Poisson Distribution}

	A random variable Y is said to have a Poisson distribution with parameter $\lambda$ if its probability is given by the probability mass function
	$$Pr(Y = y) = \frac{e^{-\lambda}\lambda^{y}}{y!}$$
	for $\lambda > 0$ and y = 0, 1, 2, ... 
 
	The mean and variance of this distribution can be shown to be
	$$E(Y) = Var(Y) = \lambda$$ 

\end{frame}



\begin{frame}[fragile]\frametitle{What is Poisson Regression?}
	In Poisson regression
	
	\begin{itemize}
	
		\item Model used when the response variable, Y, is a count (eg. Number of vehicle accidents per year, number of visits to a website over a certain time span, etc)
		\item We can also have the response variable be Y/t, the rate of the event happening with t being an interval representing time, space, or some other grouping.

	\end{itemize}

\end{frame}



\begin{frame}[fragile]\frametitle{Robust Poisson Regression}
	
	[Placeholder Text]	
	
\end{frame}



\begin{frame}[fragile]\frametitle{Smoking and Lung Cancer Dataset}

	\textbf{Smoking and Lung Cancer} 
	
	We will be using data originally collected from a Canadian study of mortality by age and smoking status
	
	The file containing cleaned data comes from the website of Professor German Rodriguez of Princeton University 
	
	link: http://data.princeton.edu/wws509/datasets/smoking.raw

\end{frame}



\begin{frame}[fragile]\frametitle{Smoking and Lung Cancer Dataset}

	\begin{figure}
	\centering
		\includegraphics[height=4cm]{"data"}
	\end{figure}

	\begin{itemize}
		\footnotesize{
			\item age at start of study: coded 1 to 9 for 40-44, 45-49, 50-54, 55-59, 60-64, 65-69, 70-74, 75-79, 80+ respectively
			\item smoking status: 1 = never smoked, 2 = smoked cigars or pipe only, 3 = smoked cigarettes and cigar or pipe, 4 = smoked cigarettes only
			\item population: number of male pensioners followed
			\item deaths: number of deaths in a six-year period
		}
	\end{itemize}

\end{frame}



\end{document}