\documentclass{beamer}
%
% Choose how your presentation looks.
%
% For more themes, color themes and font themes, see:
% http://deic.uab.es/~iblanes/beamer_gallery/index_by_theme.html
%
\mode<presentation>
{
  \usetheme{metropolis}      % or try Darmstadt, Madrid, Warsaw, ...
  \usecolortheme{default} % or try albatross, beaver, crane, ...
  \usefonttheme{default}  % or try serif, structurebold, ...
  \setbeamertemplate{navigation symbols}{}
  \setbeamertemplate{caption}[numbered]
} 

\usepackage{bookmark}
\usepackage[english]{babel}
\usepackage[utf8x]{inputenc}
\usepackage{framed}
\usepackage{amsmath,amsfonts,amssymb}
\usepackage{hyperref}
\usepackage{verbatim}
\usepackage{dsfont}
\usepackage{ulem}

\def\s{\phantom{-}}

\title[Introduction to R]{Introduction to programming in R}
\author{Dr. JC Rom{\'a}n}
\institute{SDSU Mathematics and Statistics}
\date{}



\begin{document}
	

	%\begin{frame}[fragile]
\begin{frame}[plain]
	\titlepage
	
	\footnote{a}
\end{frame}

\begin{frame}[fragile]\frametitle{What is R?}
	
	\textbf{Language:} Object-Oriented, high-level language based on \textbf{S}. Interpreted (uses scripts), similar to Python, Matlab. \\
	
	\textbf{Software:} Modular. Packages download from CRAN (easy install from inside R). Free under GNU GPL \& other public licenses.\\
	
	\textbf{Download:} \url{http://www.r-project.org/}
	
\end{frame}

\begin{frame}\frametitle{Downloading R}
	\begin{enumerate}		
		\item Visit \url{http://www.r-project.org/}
		\item Click on ``Download R"
		\item Scroll down to USA and click on \url{http://cran.stat.ucla.edu/}
		\item Download the appropriate installer for your operating system.
	\end{enumerate}
\end{frame}

\begin{frame}[fragile]\frametitle{RStudio} 
	R-Studio is an integrated development environment (IDE)/ graphical user interface (GUI) for R that adds a few useful features.
	\begin{itemize}
		\item Improved GUI, package management, coding tools: Code Completion, Syntax Highlighting, etc.
		\item Consistency across platforms: Windows, Mac OS , Linux
	\end{itemize}
	R-Studio can be downloaded for free at
	\begin{center}
		\url{http://www.rstudio.com}\\
	\end{center}
\end{frame}

\begin{frame}[fragile]\frametitle{R Tutorials}
	\textbf{Resources:}
	\begin{enumerate}
		\item \textbf{Quick-R} @ \url{http://www.statmethods.net}
		\item R-tutor @ \url{http://www.r-tutor.com/}
		\item Interactive \textbf{R} sessions via \texttt{swirl} @ \url{http://swirlstats.com/} and at \url{http://tryr.codeschool.com/}
		\item Online search 
	\end{enumerate}	
\end{frame}

\begin{frame}[fragile]\frametitle{Reading Data into R}
	\begin{itemize}
		\item We load a text file into R with \verb|read.table("filename.txt")|
		\item We load a csv file into R with \verb|read.csv("filename.csv")|
	\end{itemize}
	Example:
	\begin{framed}
		\begin{verbatim}
		> #read csv
		> nba=read.csv(file = "NBA2015Data.csv", header = T)
		\end{verbatim}
	\end{framed}
\end{frame}


\begin{frame}[fragile]\frametitle{Basic manipulation with Dataframes}
	With the nba dataset loaded, let's answer a few questions:
\vspace{-10pt}
\small
	\begin{itemize}
		\item What are the variable names?\\
		\verb|> names(nba)|\\
		\verb|"Player" "Team" ... "PLUSMINUS"|
		\item What are the dimensions of nba? 492 rows and 28 columns \\
		\verb|> dim(nba)|\\
		\verb|[1] 492  28|
		\item Show the first six rows of the dataset.\\
		\verb|> head(nba)|
		\item Let's find the statistics of nba player Lebron James.\\
		\verb|> nba[nba$Player == "LeBron James", ]|
		\item Let's find specifically his points per game (PTS) and field goal percentage (FG\%) stats.\\
		\verb|> nba[nba$Player == "LeBron James", c("PTS", "FGP")]|\\
		\verb|     PTS   FGP|\\
		\verb|     25.3  48.8|
	\end{itemize}	
\end{frame}

\begin{frame}[fragile]{Basic Manipulation with Data Frames continued}
	What are the summary statistics of PPG and FG\%?
	\begin{framed}
		\vspace{-10pt}
		\begin{verbatim}
		> summary(nba[ , c("PTS", "FGP")])
		
		PTS              FGP        
		Min.   : 0.000   Min.   :  0.00  
		1st Qu.: 4.000   1st Qu.: 39.60  
		Median : 6.900   Median : 42.90  
		Mean   : 8.123   Mean   : 43.13  
		3rd Qu.:11.625   3rd Qu.: 47.60  
		Max.   :28.100   Max.   :100.00  
		\end{verbatim}
	\end{framed}
\end{frame}

\begin{frame}[fragile]\frametitle{Regression Example}
	Plot a linear model of PTS vs MIN (Time on Court per game) for the nba dataset.
	\vspace{-5pt}
	\begin{framed}
		\verb|> plot(nba$MIN, nba$PTS,|\\
		\verb|+ xlab = "minutes", ylab = "points per game",|\\
		\verb|+ main = "NBA PPG vs MIN")|\\
		\verb|> abline(lm(PTS ~ MIN, data = nba), col = "red")|
\vspace{-18pt}
\begin{figure}
\centering
\includegraphics[width=5cm, height=5cm]{"nba plot1"}

\label{fig:nba-plot1}
\end{figure}
	\end{framed}
\end{frame}

\begin{frame}[fragile]\frametitle{Regression Example Continued}
	One could obtain a better fitting curve to the plot of PTS vs MIN with more statistical knowledge and tools as shown:
\begin{figure}
\centering
\includegraphics[width=7cm, height=7cm]{"nba plot2"}
\label{fig:nba-plot2}
\end{figure}
\end{frame}
%
%\begin{frame}[fragile]\frametitle{R vs Matlab} 
%	\begin{itemize}
%		\item 	Like Matlab, \textbf{R} is widely used as a computing tool. 
%		\item 	Syntax is very similar between R and Matlab.
%		\item	\textbf{R} excels at statistics, graphics, many packages available, free
%		\item \textbf{Matlab }is well supported and widely used.
%		\item R is open source, whereas MatLab is closed source.
%		\item 	R command "cheat sheet" for Matlab users:
%		\url{http://mathesaurus.sourceforge.net/octave-r.html}
%		\item 	R command "cheat sheet" for Matlab users:
%		\url{http://mathesaurus.sourceforge.net/octave-r.html}
%		\item 	David Hiebler's Matlab/R Reference \url{http://math.umaine.edu/~hiebeler/comp/matlabR.html}
%	\end{itemize}
%\end{frame}

%
%\begin{frame}[fragile]\frametitle{R Comments with other Computational Programs} 
%	\textbf{R} competes well against other statistical programs such as: \textbf{SAS}, \textbf{Minitab}, etc.\\
%	\begin{itemize}
%		\item \url{http://r4stats.com/articles/popularity/}
%		\item \url{http://www.analyticsvidhya.com/blog/2015/05/infographic-quick-guide-sas-python/} 
%		\item \url{http://www.burtchworks.com/2015/05/21/2015-sas-vs-r-survey-results/} \\
%	\end{itemize}
%	
%	\textbf{Python is a strong contender!} Popular in physics, engineering, web development, SAGE is python based, etc. \textbf{R} slower, but excels at statistics and graphics.\\
%	
%	\textbf{See this R vs Python comparison for details:} \url{https://www.datacamp.com/community/tutorials/r-or-python-for-data-analysis}\\
%	
%	Packages exist to run \textbf{R} code within \textbf{Python}, and \textit{vice versa}\\
%	
%\end{frame}
%
%



	
\begin{frame}[fragile]\frametitle{Defining Types}
		There are several \textit{types} used in R: numeric, logical, character, integer, matrix, array, data frame, etc.
		\begin{itemize}
			\item Numeric types are decimal value objects stored as single numbers or a vector
			\item Integer types are the integer part of a numeric type.  
			\item Logical types are logical values TRUE or FALSE.
			\item Character types are strings represented in R.
			\item Arrays are objects/vectors with multiple dimensions
			\item Matrices are two-dimensional arrays
			\item A data frame is used for storing data tables. It is a collection of vectors of equal length.
		\end{itemize}
\end{frame}

\begin{frame}[fragile]\frametitle{Example of Types}
	\begin{framed}
				\vspace{-0.25cm}
		\begin{verbatim}
		> A = 1
		> class(A)
		[1] "numeric"
		> as.integer(A) #convert to integer use
		[1] 1
		> class(as.integer(A))
		[1] "integer"
		> B = "SDSU REU" #string
		> class(B) 
		[1] "character"
		> C = 1234.5 
		> class(C)
		[1] "numeric"
		
		\end{verbatim}
	\end{framed}
\end{frame}

\begin{frame}[fragile]\frametitle{Example of Types}
	\begin{framed}
		\vspace{-0.25cm}
		\begin{verbatim}
		> D = T #equivalent to TRUE
		> class(D)
		[1] "logical"
		> E = c(1, 2, 3, 4, 5) # equivalent to  1:5
		> class(E)
		[1] "numeric"
		> f = array(1:6, c(2, 3)) # F is reserved for FALSE
		> class(f)
		[1] "matrix"
		> f
		[,1] [,2] [,3]
		[1,]    1    3    5
		[2,]    2    4    6
		\end{verbatim}
	\end{framed}
\end{frame}

\begin{frame}[fragile]\frametitle{Example of Types}
	\begin{framed}
		\vspace{-0.5cm}
		\begin{verbatim}
		> G = matrix(1:6, nrow = 2, ncol = 3, byrow = T)
		> class(G)
		[1] "matrix"
		> G
		[,1] [,2] [,3]
		[1,]    1    2    3
		[2,]    4    5    6
		> H = as.matrix(f)
		> class(H)
		[1] "matrix"
		> H
		[,1] [,2] [,3]
		[1,]    1    3    5
		[2,]    2    4    6
		\end{verbatim}
	\end{framed}
\end{frame}

\begin{frame}[fragile]\frametitle{Fibonacci Example}
	Recall the Fibonacci sequence is defined recursively from $F(n)=F(n-1)+F(n-2)$ for $n>1$ with $F(0)=0$ and $F(1)=1$.
	\begin{framed}
		\begin{verbatim}
		> fibR <- function(n){
		+   if(n == 0){return(0)}
		+   if(n == 1){return(1)}
		+   return(fibR(n-1) + fibR(n-2))}
		> start.time <- Sys.time()
		> fibR(40)
		[1] 102334155
		> Sys.time() - start.time #How long to compute
		Time difference of 6.22482 mins 
		\end{verbatim}
	\end{framed}
	
	\textbf{Remark}:  There are more efficient ways to code this function. 
\end{frame}


\begin{frame}[fragile]\frametitle{Fibonacci Example Continued}
	Let's make a plot of the time it takes to compute the $n^{th}$ Fibonacci number versus the index.
	\begin{framed}
		\begin{verbatim}
		> n = 1:30
		> time = rep(NA, length=30)
		> for(i in 1:30){
		+   start.time = Sys.time()
		+   fibR(i)
		+   time[i] = Sys.time() - start.time
		+ }
		> plot(n,time,type = "l")
		\end{verbatim}
	\end{framed}
\end{frame}

\begin{frame}{Fibonacci Example Continued}
	
\begin{figure}
\centering
\includegraphics[width=7cm, height=7cm]{"Calc time vs Fibonacci plot"}
\label{fig:calc-time-vs-fibonacci-plot}
\end{figure}
		
\end{frame}


\begin{frame}[fragile]\frametitle{Pros and Cons of R}
	\textbf{R Pros}?
	\begin{enumerate}
		\item \textbf{R} for \textbf{statistics}, or as a \textbf{general computing platform}
		\item \textbf{Free} and widely used in academia and industry
		\item Many resources to support \textbf{teaching} and \textbf{research}
		\item Integrates well with other software
	\end{enumerate}
	\textbf{Cons?}
	\begin{enumerate}
		\item \textit{Slow!} Not a low-level language
		\item Symbolic tools are limited 
		\item Memory management problems (depending on your OS), especially when displaying high resolution images or working with huge matrices (hundreds of Mb).
		\item \textbf{Learning Curve!}\\ R is lower than specialized ``point-and-click" tools
		
	\end{enumerate}
\end{frame}

\begin{frame}[fragile]\frametitle{R packages}
	\begin{itemize}
		\item To install new custom packages/libraries use \verb|install.packages("packagename")|.
		\item To load the package into R use \verb|library(packagename)|.
		\item Parallel computing requires ``doParallel" package
		\item C++ integration requires packages ``Rcpp" and ``inline".
%		\item JAGS (for MCMC algorithms) integration requires ``Rjags" package.
	\end{itemize}
\end{frame}
%
%\begin{frame}[fragile]\frametitle{Extensions}
%	\begin{itemize}
%		\item Fancy graphics: ggplot, shiny, plotly, etc.
%		\item Online compiler: \url{www.getdatajoy.com}
%		\item Other Programming tool: Julia @ \url{http://julialang.org/}
%		\begin{itemize}
%			\item Julia is a high-level, high-performance dynamic programming language for technical computing, with syntax that is familiar to users of other technical computing environments like R or MatLab.
%			\item Performance is close to C/C++.
%			\item Tutorials @ 
%			\begin{itemize}
%			\item \url{http://forio.com/labs/julia-studio/tutorials/} \item \url{http://www.quant-econ.net/jl/index.html}
%			\end{itemize}
%		\end{itemize}
%	\end{itemize}
%\end{frame}
\begin{frame}\frametitle{Exercises}
	\begin{enumerate}
		\item Create a vector that samples 15 numbers from the sequence 1,2,3,...20; once with replacement, and once without replacement. (Hint: use "sample()")
		\item Find the mean and standard deviation of the sample. (use "mean()",  "sqrt(var())", and "sd()")
		\item Plot a histogram of the sample. (Use the "hist()" function)
		\item Plot the function $f(x)=x^3+3x^2+3x+1$ using R. Title the plot with the function $f(x)=x^3+3x^2+3x+1$. Add color. 
		\item Create a function that considers numerical inputs and returns the 1 if the input is odd, and 0 if the input is even. 
		\item Create a function that computes the inverse of a 2x2 matrix. Compare your function with ``solve(matrix)".
		\item Create a function that computes the $n^{th}$ term of the sequence as defined by $a_n=a_{n-1}+a_{n-2}-a_{n-3}$ for $n>2$ where $a_0=1, a_1=2,a_2=3$.
	\end{enumerate}
\end{frame}
\end{document}